\subsection{Elaborarea lucr\u{a}rii}
La sarcina cu matricile am definit o clasa cu 2 atribute private: pointer 
la pointer(pentru a lucra cu matricea propriu zis) \c{s}i un parametru 
care indic\v{a} dimensiunea matricii, strict necesar la opera\c{t}ia 
\^{i}ntre matrici. Nu este posibil de adunat/sc\v{a}zut dou\v{a} matrici 
de dimensiuni diferite, \c{s}i pentru a evita segmentare de memorie sau alte
probleme - am introdus verificarea compatibilit\v{a}\c{t}ii.

Am ad\v{a}ugat metodele necesare sarcinii prin overload la operatori:\\
\texttt{matrix\& operator+(matrix\& B);}, constructori(f\v{a}r\v{a} parametri,
cu parametru), metode de intrare \c{s}i ie\c{s}ire pentru a vedea mai
u\c{s}or rezultatul execu\c{t}iei.\\

Pentru Java sarcina a fost simpl\v{a}. Am creat dou\v{a} clase - una pentru
lucrul cu lista de numere intregi, \c{s}i o clasa cu punctul de intrare, 
similar la \textbf{main()} din C++. Deoarece ambele clase sunt \^{i}n 
acela\c{s}i folder - nu este nevoie de importat explicit clasa creat\v{a},
ea se afla \^{i}n acela\c{s}i classpath, dar pentru proiecte mai mari ar
fi nevoie de indicat package-urile, import-urile din alte package-uri etc.
Ca verificare a metodei \textbf{toString()} am rulat metoda de 2 ori - 
odat\v{a} metoda predefinit\v{a} pe clasa \textbf{ArrayList}, \c{s}i 
metoda redefini\v{a} pe clasa creata \textbf{Lista}.

Rezultatul execu\c{t}iei programului Java:
\begin{verbatim}
[2, 9, 4, 5, 7, 8]
{2, 9, 4, 5, 7, 8}
\end{verbatim}

Rularea programului C++:
\begin{verbatim}
Introdu dimensiunea matricelor:3
Introdu prima matrice
A[0][0]:1
A[0][1]:2
A[0][2]:3
A[1][0]:4
A[1][1]:5
A[1][2]:6
A[2][0]:7
A[2][1]:8
A[2][2]:9
Introdu a doua matrice
A[0][0]:2
A[0][1]:3
A[0][2]:6
A[1][0]:5
A[1][1]:7
A[1][2]:8
A[2][0]:9
A[2][1]:1
A[2][2]:2
Ati introdus:
    1   2   3
    4   5   6
    7   8   9
a doua matrice:
    2   3   6
    5   7   8
    9   1   2
suma matricelor:
    3   5   9
    9   12  14
    16  9   11
diferenta matricelor:
    -1  -1  -3
    -1  -2  -2
    -2  7   7
transpusa primei matrici:
    1   4   7
    2   5   8
    3   6   9
\end{verbatim}

